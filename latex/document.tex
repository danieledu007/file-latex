\documentclass[12pt]{article}
\author{Daniel,Youseff}
\title{Esempio della regressione lineare usando l'andamento del prezzo degli stock di nvidia}
\begin{document}
\maketitle
\section{cenni teorici}
La regressione lineare è un metodo statistico utilizzato per modellare la relazione tra una variabile dipendente
\textbf{Y} (detta anche risposta o output) e una o più variabili indipendenti 
\textbf{X} (dette anche predittori o input). Lo scopo principale è stimare una funzione lineare che descriva come \textbf{Y} 
varia al variare di \textbf{X}.
\\Nel caso più semplice (regressione lineare semplice, con una sola variabile indipendente), il modello si esprime come:
\begin{equation}
    Y=\beta_0+\beta_1*X+\varepsilon  
\end{equation}
\begin{itemize}
    \item \textbf{Y}: variabile dipendente
    \item \textbf{X}:variabile indipendente
    \item \begin{math}\mathbf{\beta_0}\end{math}: intercetta (il valore previsto di \textbf{Y}quando \textbf{X}=0)
    \item \begin{math} \mathbf{\beta_1}\end{math}: coefficiente angolare (indica quanto cambia \textbf{Y} per ogni unità
          di incremetno in \textbf{X})
    \item 
\end{itemize}
\end{document}